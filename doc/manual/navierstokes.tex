%
%
% SUMMARY:
% USAGE:
%
% AUTHOR:       Gilles Fourestey
% ORG:          EPFL
% E-MAIL:       foureste@iacspc.epfl.ch
%
% ORIG-DATE:  4-Feb-09 at 14:11:50
% LAST-MOD:  5-Feb-09 at 18:11:24 by Gilles Fourestey
%
% DESCRIPTION:
% DESCRIP-END.


Now that we have decribed a simple stationary problem, let's have a look at the evolutionary
case. In this example, we will consider the same domain, but this time we will solve the
Navier-Stokes problem \ixn{Navier-Stokes Problem}. Starting from the Oseen problem (\ref{eqn-oseen})

\begin{equation*}
\left\{
\begin{array}{rl}
\displaystyle \alpha u + \beta \cdot \nabla u - \nu \Delta u+
\nabla p & = f \\
\displaystyle \nabla \cdot u & = 0
\end{array}
\right.
\end{equation*}

We want to solve the non-stationary Navier-Stokes problem

\begin{equation*} \label{eqn-navierstokes}
\left\{
\begin{array}{rl}
\displaystyle \frac{\partial u}{\partial t} + u \cdot \nabla u - \nu \Delta u+
\nabla p & = f \\
\displaystyle \nabla \cdot u & = 0
\end{array}
\right.
\end{equation*}

which can be written, using a semi-discretization of the time partial derivative as

\begin{equation*} \label{eqn-nsnl}
\left\{
\begin{array}{rl}
\displaystyle \alpha u^{n+1} + u^{n+1} \cdot \nabla u^{n+1} - \nu \Delta u^{n+1}+
\nabla p^{n+1} & = f^n  \\
\displaystyle \nabla \cdot u & = 0  \\
\end{array}
\right.
\end{equation*}

Where $\alpha$ is a constant which depends on the time discretization order, 
$\Delta t$ is the time step, $u^{n+1}$ and $p^{n+1}$ 
are the velocity and the pressure at the time $t^n = n\Delta t$.
Note that $f^n$ now contains terms resulting from the time discretization.
Using a linearization $\beta^n$ of (\ref{eqn-nsnl}) around $u^{n+1}$, 
we get the full semi-discrete linearized Navier-Stokes
equations

\begin{equation*} \label{eqn-ns}
\left\{
\begin{array}{rl}
\displaystyle \alpha u^{n+1} + \beta^n \cdot \nabla u^{n+1} - \nu \Delta u^{n+1}+
\nabla p^{n+1} & = f^n  \\
\displaystyle \nabla \cdot u & = 0  \\
\end{array}
\right.
\end{equation*}

Solving (\ref{eqn-ns}) using the framework we used for the stationary driven cavity is not difficult.
Instead of setting $\alpha$, $\beta$ and $f^n$ to zero, we give them their proper values. For instance,
consider the first order discretization in time

\begin{equation*} \label{eqn-nso1}
\left\{
\begin{array}{rl}
\displaystyle \frac {u^{n+1}}{\Delta t} + u^n \cdot \nabla u^{n+1} - \nu \Delta u^{n+1}+
\nabla p^{n+1} & = \displaystyle \frac{u^n}{\Delta t}  \\
\displaystyle \nabla \cdot u & = 0. \\
\end{array}
\right.
\end{equation*}

Or the second order discretization 

\begin{equation*} \label{eqn-nso2}
\left\{
\begin{array}{rl}
\displaystyle \frac {3u^{n+1}}{2\Delta t} + u^n \cdot \nabla u^{n+1} - \nu \Delta u^{n+1}+
\nabla p^{n+1} & = \displaystyle \frac{2u^n}{\Delta t} - \frac{u^{n-1}}{2\Delta t}  \\
\displaystyle \nabla \cdot u & = 0. \\
\end{array}
\right.
\end{equation*}

Let's have a look at the code in \verb!testsuite/test_cavity/main.cpp!.
Until the first \verb!fluid.iterate()!, the code is the same as the one used to compute the Stokes problem.
Now, we want to use this Stokes solution to initialize our non-stationary Navier-Stokes problem, and be able
to store a history of previous solutions in order to compute the time derivative.
This can be done by using the following object 
\begin{verbatim}
    // bdf initialization with the stokes problem solution
    BdfTNS<vector_type> bdf(oseenData.getBDF_order());
\end{verbatim}
The backward differentiation formula, class \verb!BdfTNS! \ixn{Backward Differentiation Formula}, 
stores the previous solution $u^n$, $u^{n-1}$ ... 
All we need to do is to construct this class
with the correct time discretization order given in the data file (variable \verb!fluid/discretization/bdf_order!) 
so that the stored vector will be resized correctly, and intialize it with our previously computed Stokes problem solution

\begin{verbatim}
   bdf.bdf_u().initialize_unk( fluid.solution() );
\end{verbatim}

We are now ready to enter the time loop 

\begin{verbatim}
    Real dt     = oseenData.getTimeStep();
    Real t0     = oseenData.getInitialTime();
    Real tFinal = oseenData.getEndTime ();

    int iter = 1;

    for ( Real time = t0 + dt ; time <= tFinal + dt/2.; time += dt, iter++)
    {
        // inside the time loop, it's really like the initialization procedure,
        // exept that we now have an advection velocity, rhs and the mass matrix
        oseenData.setTime(time);

        if (verbose)
        {
            std::cout << std::endl;
            std::cout << "We are now at time " << oseenData.time()
                      << " s. " << std::endl;
            std::cout << std::endl;
        }

        chrono.start();

        // alpha coefficient for the mass matrix
        double alpha = bdf.bdf_u().coeff_der( 0 ) / oseenData.timestep();

        // extrapolation of the advection term
        beta = bdf.bdf_u().extrap();

        // rhs  part of the time-derivative
        rhs  = fluid.matrMass()*bdf.bdf_u().time_der( oseenData.timestep() );

        // update the Oseen system
        fluid.updateSystem( alpha, beta, rhs );

        // and we solve it
        fluid.iterate( bcH );

        // shifting the previous solutions
        bdf.bdf_u().shift_right( fluid.solution() );

        // postprocess
        *velAndPressure = fluid.solution();
        ensight.postProcess( time );

        // a barrier for sinchronization 
        MPI_Barrier(MPI_COMM_WORLD);

        chrono.stop();
        if (verbose) std::cout << "Total iteration time "
                               << chrono.diff()
                               << " s." << std::endl;
    }

\end{verbatim}

The time step \verb!dt!, the initial time \verb!t0! and the final simulation time \verb!tFinal!
are found in the data file, their names are respectively  \verb!fluid/discretization/timestep!,
\verb!problem/Tstart! and \verb!fluid/physics/endtime!. As we can see, a time step can be described
as a follow up of several intuitive calls:
\begin{itemize}
\item computation of $\alpha$ (which should is constant in most cases),
\item computation of $\beta$ using the \verb!Bdf! class,
\item computation of the right hand side $rhs$,
\item update of the system using these three variables,
\item solution of the linear system.
\end{itemize}
After the system is solved, we simply update all time-dependent variables such as the storage
of the previous solutions and we loop until all time steps are computed.


%
%%%%%%%%%%%%% Some Settings for emacs and auc-TeX
% Local Variables:
% TeX-master: t
% TeX-command-default: "PDFLaTeX"
% TeX-parse-self: t
% TeX-auto-save: t
% x-symbol-8bits: nil
% TeX-auto-regexp-list: TeX-auto-full-regexp-list
% eval: (ispell-change-dictionary "american")
% End:
%
