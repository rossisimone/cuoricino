\section{Continuum mechanics and Constitutive laws} This section deals
with the derivation and formulation of the conservation equation of
linear momentum and the description of the constitutive laws available
in \LV.

\subsection{Continuum mechanics}
\label{sct-Continuum} In order to describe the motion and deformations
of a body, we embed it in a three-dimensional Euclidean space. Let
\RefCon $\subseteq$ \Real be the \textit{reference configuration} of
the body. This enables us to uniquely identify any arbitrary point
$P(\posE)$, called \textit{material point}, in \RefCon by its position
at the initial time $t_0=0$. When the body moves, at each time $t>0$,
it will occupy a new configuration \CurCon, which will be called the
\textit{current configuration}. In the study of the kinematics of a
body, it is commonly assumed that the motion of an arbitrary material
point, starting from \pos, can be described through a relationship of
the form:
\begin{equation}
  \nposE=\chi(\posE,t)\qquad t\geq0,
  \label{eq::deformation}
\end{equation}
where \npos is the new position of the point \pos,
$\chi$ is a continuously differentiable function in all variables (at
least up to the second derivatives). Moreover, we assume that at each
time $t>0$, the following property on $\chi$ holds: for any \pos and
corresponding point \npos, there are open balls $B_{\posE}$ (centered
in \pos) and $B_{\nposE}$ (centered in \npos), both contained in
\RefCon and \CurCon, such that points in $B_{\posE}$ are in one-to-one
correspondance with points of $B_{\nposE}$.\\ Thanks to the map
\eqref{eq::deformation} we can define the displacement of a point
$P(\posE,t_0=0)$ at time $t>0$ as
\begin{equation}
  \displL=\nposE(\posE,t)-\posE
  \label{eq::displacementL}
\end{equation}
It is worth noting that the displacement field has
been defined in the Lagrangian framework. Its Eulerian description is
given by:
\begin{equation}
  \displE=\nposE-\chi^{-1}(\nposE,t).
  \label{eq::displacementE}
\end{equation}
According to these two representations, it is possible
to define other kinematics quantities (e.g. velocity and acceleration
fields) in each of the two coordinate systems.\\ The local deformation
of a material point is characterized by a second order tensor, usually
indicated by \F and called \textit{deformation gradient}, which is
defined as:
\begin{equation}
  \F = \delOp\nposE,
  \label{eq:tensorF}
\end{equation}
where the symbol $\delOp$ indicates that the
derivatives are computed with respect to the coordinates in the
reference configuration. In addition, the tensor \F{} allows defining
other classical tensorial quantities, like the right Cauchy-Green and
Green-Lagrange tensors, to quantify the stretches and strains during
the deformation.\\

\subsubsection{Conservation equation of linear momentum}
\label{sct-Conservation} The postulate of balance of linear momentum
states that the rate of change of linear momentum of a fixed mass of
the body is equal to the sum of the forces acting on the body. These
can be body forces or forces due to stress vectors acting on the
surface of the body,
\begin{equation}
  \frac{d}{dt}\int_{\CurConE}\rho\velE dv =
  \int_{\CurConE}\rho\underline{b}
  dv+\int_{\partial\CurConE}\underline{t}da
  \label{eq::BLM-IntegralFormEulerian}
\end{equation}
where $\rho$ is the density of the material in the
current configuration (and it is related to its counterpart in \RefCon
thanks to $\rho=J\rho_0$, where $J$ is the determinant of the deformation
gradient \F) and $\velE$ is the Eulerian velocity. The
vector $\underline{b}$ is the body force per unit of mass,
$\underline{t}=\underline{t}(\nposE,t, \underline{n})$ is the surface
force acting on the body in the current configuration per unit area of
$\partial\CurConE$ and $\underline{n}$ is the outward unit normal to
the surface $\partial\CurConE$ at \npos at time $t$. The first and the
second integrals on the right hand side represent the contributions
due to body forces and to surface forces, respectively.\\ The surface
tension $\underline{t}$, thanks to the Cauchy lemma and the relation
between the stress tensor and the stress vector, can be written as:
\begin{equation}
  \underline{t}=\underline{t}(\nposE,t,
  \underline{n})=\T(\nposE,t)\underline{n}.
  \label{eq::tension}
\end{equation}
Substituting \eqref{eq::tension} in
\eqref{eq::BLM-IntegralFormEulerian} and using the divergence theorem,
equation \eqref{eq::BLM-IntegralFormEulerian} becomes:
\begin{equation}
  \frac{d}{dt}\int_{\CurConE}\rho\velE dv =
  \int_{\CurConE}\big(\rho\underline{b} +\text{div}\T \big)dv.
  \label{eq::BLM-Eulerian}
\end{equation}
Equation \eqref{eq::BLM-Eulerian} is the Eulerian
integral form of the conservation equation of linear momentum. In
order to write it in its Lagrangian form we need to compute the
integrals on the reference configuration \RefCon. Using the identities
$dv=JdV$, $\rho=J\rho_0$, $da=J\F^{-T}dA$, where $J=\text{det}\F$, and
remembering the definition of the First Piola-Kirchhoff tensor
($\Piola=J\T\F^{-T}$), the Lagrangian form of
\eqref{eq::BLM-IntegralFormEulerian} is:
\begin{equation}
  \frac{d}{dt}\int_{\RefConE}\rho_0\velL dV =
  \int_{\RefConE}\big(\rho_0\underline{b}+\text{Div}\Piola \big)dV,
\end{equation}
where $\velL$ is the Lagrangian velocity and
$\text{Div}(\cdot)$ is the divergence operator in the material
coordinates. Assuming a constant density both in space and time, the
local Lagrangian form of the conservation equation of linear momentum
is:
\begin{equation}
  \rho_0\frac{\partial^2\displL}{\partial
    t^2}=\rho_0\underline{b}+\text{Div}(\Piola) \qquad \forall \posE
  \subseteq \RefConE, \quad t>0,
  \label{eq::ConLinMom-Diff}
\end{equation}
which is then augmented with proper initial and boundary conditions.
Subsequentely, the mechanical behaviour of the material is characterized
by specifying a particular constitutive law for the Cauchy stress tensor.

\subsection{Constitutive laws}
\label{sct-Constitutive} Equation \eqref{eq::ConLinMom-Diff} must be
closed by an adequate constitutive law in order to have a solveable
mathematical problem. The closure equation relates the Cauchy stress
tensor \T to the deformation gradient \F. The most general form of
this kind of relation is:
\begin{equation}
  \T(\nposE)=\mathcal{G}\big(F(\nposE,t),\nposE\big).
  \label{eq::GeneralCL}
\end{equation}
In the following we will consider Cauchy elastic
materials (i.e. materials in which the current stresses in \CurCon are
determined only by the current state of deformation) and homogeneous
materials (i.e. the material behaviour is independent of the material
point). Under these hypothesis, relation \eqref{eq::GeneralCL} reduces
to the form
\begin{equation}
  \T(\nposE)=\mathcal{G}\big(F(\nposE,t)\big),
  \label{eq::OurRelation}
\end{equation} where $\mathcal{G}$ is called the response function.\\
The structural models which have been implemented in the \SSol
framework are for isotropic hyperelastic materials. A material is said
to be hyperelastic when it does not dissipate energy during cyclic
homogeneous deformations,
\begin{equation}
  \text{W}_{\text{cycle}} =
  \int_{0}^{\text{T}}\int_{\RefConE}\,\mathbf{P}:\dot{\mathbf{F}}\,
  d\RefConE\text{dt}
  = 0,
  \label{eq::energyDiss}
\end{equation}
along any deformation characterized by
$\posE(t=T)=\posE(t=0)$ at any point of \RefCon. For hyperelastic
materials, the stress power can be represented in the following way:
\begin{equation}
  \rho\frac{D\WE(\posE,t)}{Dt}=\T:\D
  \label{eq::HyperSP}
\end{equation}
where \D is the velocity gradient tensor and \W is the
\textit{strain energy density function}. This function characterizes
one particular material from another. From the mathematical point of
view, the strain energy function of an isotropic hyperelastic material
is a function of the right Cauchy-Green tensor ($\C=\F^{T}\F$) and, in
particular, without loss of generality, \W can be written in terms of
the invariants of \C,
\begin{equation}
  \WE=\WE\big(I_{\C},II_{\C},III_{\C}\big).
  \label{eq::WandInvariants}
\end{equation}
All the tensors (e.g. \F,\T,\Piola) that describe the
mechanical response of this kind of materials can be computed by
derivation of \W with respect to $I_{\C},II_{\C},III_{\C}$ and \C.This
theory will not be discussed in depth in this report, but a more
detailed explanation can be found in \cite{GM,Deluca} and references
therein. We would, however, like to remind that in order to guarantee
the well posedness of the structural mechanics problem
\eqref{eq::ConLinMom-Diff} the strain energy function \W of the
material has to satisfy the condition of policonvexity.\\ In the
following we will characterize the strain energy function for all the
four stress-strain relations that are available in \LV and we will
give the corresponding expression of the first Piola-Kirchhoff
tensor. In particular, the first two models that we analyze describe
isotropic hyperelastic compressible materials and the second two
isotropic hyperelastic nearly-incompressible materials. Since the
applications in which we are interested concern models of incompressible
materials, we will enforce the incompressibility constraint in our
simulations:
\begin{itemize}
\item by using high Poisson ratios for the first two
  models\footnote{The choice of high Poisson ratios may be critical in
    terms of a well-known problem: locking [T. Hughes, \textit{The finite element method:
        Linear Static and Dynamic Finite Element Analysis} - Englewood Cliffs - NJ : Prentice-Hall, 1987  ]}
\item by recovering the common multiplicative decomposition of the
  deformation gradient \F into an isochoric and volumetric part for the
  second two (see \cite{BonetWood}).
\end{itemize} In the following subsections, the computations for the
definition of the first Piola-Kirchhoff tensor will not be detailed,
but can be found in \cite{Deluca} and references therein.

\subsubsection{St. Venant-Kirchhoff and Linear Elastic models} Both of
these models take origin from the same strain energy function. The
characteristic function \W for these two constitutive laws is:
\begin{equation}
  \WE\big(\Et\big)=\frac{\lambda}{2}(\text{tr}\Et)^2 +
  \mu\Et:\Et
  \label{eq::W-SVKLE}
\end{equation}
where \Et is the Green-Lagrange tensor and it is
defined as $\Et=\frac{1}{2}\big(\C-\textbf{I}\big)$, and $\lambda$ and
$\mu$ are the first Lam\'e constant and the shear modulus
respectively. These two materials parameters are functions of the
Young modulus E and Poisson ratio $\nu$ as follows:
\begin{equation}
  \lambda=\frac{\nu E}{\big(1+\nu \big)\big(1-2\nu
    \big)}\qquad\qquad \mu=\frac{E}{2\big(1+\nu\big)}.
  \label{eq::LameConst}
\end{equation}
Since the material is hyperelastic and isotropic, as we
said earlier, \W can be written in terms of the invariants of \C,
leading to the relation:
\begin{equation}
  \WE=\left(\frac{\lambda}{8}+\frac{\mu}{4}\right)I_{\C}^2 -
  \left(\frac{3\lambda}{4}+\frac{\mu}{2}\right)I_{\C} -
  \frac{\mu}{2}II_{\C}+\frac{9\lambda}{8}+\frac{3\mu}{4}
  \label{eq::WSVK-Inv}
\end{equation}
For the St. Venant-Kirchhoff material the first
Piola-Kirchhoff tensor is:
\begin{equation}
  \Piola = \frac{\lambda}{2}\big(I_{\C}-3\big)\F +
  \mu\F +\mu\F\C
  \label{eq::SVK-P}
\end{equation}
It is convenient to introduce the displacement field,
$\displL$, which transforms the expression \eqref{eq::SVK-P} into:
\begin{equation}
  \begin{array}{llll} \Piola(\displL) = & \displaystyle
    \lambda(\diver{\displL})\I + \mu\big(\delOp\displL +
    \delOp\displL^T\big)
    +\frac{\lambda}{2}\big(\delOp\displL:\delOp\displL\big)+ \\ \\ &
    \displaystyle + \lambda\big(\diver{\displL}\big)\delOp\displL +
    \frac{\lambda}{2}\big(\delOp\displL:\delOp\displL\big)\delOp\displL +
    \mu\big(\delOp\displL^T \delOp\displL\big)+\\ \\ & \displaystyle +
    \mu\delOp\displL\big(\delOp\displL\delOp\displL^T\big) +
    \mu\delOp\displL\delOp\displL^T\delOp\displL.
  \end{array}
  \label{eq::SVK-P-displ}
\end{equation}
We insert \eqref{eq::SVK-P-displ} in \eqref{eq::ConLinMom-Diff} when we
want to solve the structural
dynamics problem using the St. Venant-Kirchhoff constitutive law.\\ The
linear elastic model can be deduced from \eqref{eq::SVK-P-displ}
considering only the linear terms. In this case, the tensor \Piola is:
\begin{equation}
  \Piola(\displL) = \lambda(\diver{\displL})\I +
  \mu\big(\delOp\displL + \delOp\displL^T\big).
  \label{eq::LE-P}
\end{equation}
It is worth noting that the St. Venant-Kirchhoff
strain energy function \eqref{eq::W-SVKLE} does not satisfy the
policonvexity condition for all the state of deformations. For this
reason, it is not recommended when large compressive strains
occur. More details can be found in \cite{Hozapfel}.

\subsubsection{Neohookean model} The expression of \W in terms of the
invariants for the Neo-Hookean material is:
\begin{equation}
  \WE=\frac{\mu}{2}\big(I_{\Cbar}-3\big)+\frac{\kappa}{4}\left[\big(J-1)^2
    + (\text{ln}J)^2\right],
  \label{eq::strainNH}
\end{equation}
where $I_{\Cbar}$ is the trace of the isochoric part of
\C, $\mu$ is the shear modulus, $\kappa$ is the bulk modulus, and $J$
is the determinant of \F. In this case, the tensor \Piola reads:
\begin{equation}
  \begin{array}{lll} \Piola = & \displaystyle \mu
    J^{-\frac{2}{3}}\left(\F-\frac{1}{3}I_{\Cbar}\F^{-T}\right) +\\ \\ &
    \displaystyle +
    J\frac{\kappa}{2}\left(J-1+\frac{1}{J}\text{ln}J\right)\F^{-T}.
  \end {array}
  \label{eq::NH-P}
\end{equation}
The first term in Eq. \eqref{eq::NH-P} representes the
isochoric part of \Piola, while the second is the volumetric
part. This is due to the multiplicative decomposition of
\F. Introducing the displacement field, the expression above becomes:
\begin{equation}
  \begin{array}{lll} \Piola(\displL) = & \displaystyle \mu
    J^{-\frac{2}{3}}\left(\delOp\displL -
      \frac{1}{3}\big(\delOp\displL:\delOp\displL+2\diver{\displL}+3\big)
      \delOp\displL^{-T}\right)+\\
    \\ & \displaystyle +
    J\frac{\kappa}{2}\left(J-1+\frac{1}{J}\text{ln}J\right)\delOp\displL^{-T},
  \end{array}
  \label{eq::NH-P-displ}
\end{equation}
where we have used the identity:
\begin{equation} I_{\Cbar}=\text{tr}\Cbar=\delOp\displL:\delOp\displL
  + 2\diver{\displL}+3.
  \label{eq::Id-trC-displ}
\end{equation}
The Neohookean material satisfies the policonvexity
condition for all the states of deformations. Hence, the mathematical
problems generated by Eq. \eqref{eq::ConLinMom-Diff} is well posed.

\subsubsection{Exponential model} In this case, the relation between
\W and the invariants of the isochoric part of \C is the following:
\begin{equation}
  \WE= \displaystyle
  \frac{\alpha}{2\gamma}\left(e^{\gamma(\I_{\Cbar}-3)}-1\right) +
  \frac{\kappa}{4}\left[\big(J-1)^2 + (\text{ln}J)^2\right],
  \label{eq::strainExp}
\end{equation}
where $\alpha$ and $\gamma$ are material parameters and
$\kappa$ is, as before, the bulk modulus. After some computations, the
expression of the first Piola-Kirchhoff tensor is:
\begin{equation}
  \begin{array}{lll} \Piola = & \displaystyle \alpha
    J^{-\frac{2}{3}}\left(\F-\frac{1}{3}I_{\Cbar}\F^{-T}\right)
    e^{\gamma\big(I_{\Cbar}-3\big)}
    +\\ \\ & \displaystyle +
    J\frac{\kappa}{2}\left(J-1+\frac{1}{J}\text{ln}J\right)\F^{-T}.
  \end {array}
  \label{eq::EXP-P}
\end{equation}
Introducing in \eqref{eq::EXP-P} the displacement
field, the expression of \Piola as a function of $\displL$ is
obtained. The first term in Eq. \eqref{eq::EXP-P} is the isochoric
contribution and the second one the volumetric. As for the Neohookean
law, the policonvexity condition is satisfied for all the state of
deformations leading to the well posedness of the mathematical problem
\eqref{eq::ConLinMom-Diff}.
